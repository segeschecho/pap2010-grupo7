\chapter{Problema 4}

\section{Minimal Coverage}

Dados segmentos de una l�nea (en el eje X) con coordenadas [$L_i$,$R_i$]. Se debe elegir la minima cantidad de estos segmentos, tales que cubran completamente el segmento [$0$,$M$].

\textbf{Input:}

El input esta dividido en varias partes:

La primera l�nea, consiste en un n�mero e indica la cantidad de casos de test, seguido de una l�nea en blanco.

Luego sigue el test, que consiste en un n�mero, que identifica el segmento [$0$,$M$] (donde $1<= M <= 5000$), seguido de pares L R que identifican segmentos ($L_i$, $R_i$ $<=50000$, e $i <= 100000$), cada uno separados por una linea. Cada test termina con el par 0 0.

\textbf{Output:}

Para cada caso de test, la primera l�nea indica el m�nimo numero de segmentos que pueden cubrir el segmento [$0$,$M$]. En las l�neas siguientes, las coordenadas de los segmentos, ordenados por su coordenada izquierda ($L_i$), con el mismo formato que en el input. El par 0 0 no debe ser mostrado.

Si [$0$,$M$] no puede ser cubierto, se debe devolver 0.

Cada resultado entre tests, debe estar separado por una l�nea en blanco.

\textbf{Url:}

\href{http://uva.onlinejudge.org/index.php?option=com\_onlinejudge\&Itemid=8\&category=12\&page=show\_problem\&problem=961}{Problema de minimal coverage}