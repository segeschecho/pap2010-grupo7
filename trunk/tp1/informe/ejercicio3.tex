\chapter{Problema 3}

\section{Distinct Subsequences}

Una subsecuencia de una secuencia dada es la secuencia con algunos elementos (posiblemente ninguno) faltantes. Formalmente, dada una secuencia $X = x_1, x_2, ..., x_m$, otra secuencia $Z = z_1, z_2, ..., z_k$ es una subsecuencia de X si existe una secuencia estrictamente creciente $< i_1, i_2, ..., i_k >$ de �ndices de X tal que para todo $j = 1, 2, ..., k$ tenemos $x_{i_j} = z_j$. Por ejemplo, $Z = bcdb$ es una subsecuencia de $X = abcbdab$ con su correspondiente secuencia de indices \verb|< 2, 3, 5, 7 >|.

Escribir un programa tal que cuente el n�mero de subsecuencias Z de X tal que cada una de ellas tiene una correspondiente subsecuencia de indices distinta.

\textbf{Entrada:}

La primer l�nea del input contiene un entero N indicando la cantidad de casos de test que siguen.

La primer linea de cada caso de test contiene una cadena X de una longitud no mayor a 10000, compuesta enteramente de caracteres del alfabeto en min�scula. La segunda l�nea contiene otra cadena Z de longitud no mayor a 100, y tambi�n compuesta solamente de caracteres del alfabeto en min�scula. Se asegura que ni Z ni ningun prefijo o sufijo del mismo tendr� m�s de 10100 distintas ocurrencias en X como subsecuencia.

\textbf{Salida:}

Para cada caso de test de la entrada, debe haber una salida correspondiente al numero de ocurrencias de Z en X como subsecuencia. La salida de cada caso de test de entrada debe hacerse en l�neas separadas.

\textbf{Url:}

\href{http://uva.onlinejudge.org/index.php?option=com_onlinejudge\&Itemid=8\&category=12\&page=show\_problem\&problem=1010}{Problema de Distinct Subsequences}

\subsection{C�lculo de complejidad}

\begin{algorithm}[H]
\caption{Calcula la cantidad de ocurrencias de una subsecuencia en una secuencia}
\label{alg:algoritmo1b}
\begin{algorithmic}[1]
\PARAMS{Secuencia, Subsecuencia}
\STATE inicializar las Combinaciones para cada letra de la subsecuencia en 0
\STATE longSubsec $\leftarrow$ largo de Subsecuencia
\FOR{cada caracter C de la secuencia de fin a principio}
	\FOR{cada caracter CSub de la subsecuencia tal que CSub $=$ C}
		\IF CSub es el �ltimo caracter de la Subsecuencia
			\STATE Combinaciones_{longSubsec - 1} $\leftarrow$ Combinaciones_{longSubsec - 1} + 1
		\ELSE
			\STATE i $\leftarrow$ indice del caracter CSub de la subsecuencia
			\STATE Combinaciones_i $\leftarrow$ Combinaciones_{i + 1}
		\ENDIF
	\ENDFOR
\ENDFOR
\RETURN Combinaciones_0
\end{algorithmic}
\end{algorithm}