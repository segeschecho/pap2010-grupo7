 %%	SECCION documentclass																									 %%	
%%---------------------------------------------------------------------------%%
\documentclass[a4paper]{report}

%%---------------------------------------------------------------------------%%
%%	SECCION usepackage																											 %%	
%%---------------------------------------------------------------------------%%
\usepackage{amsmath, amsthm}
\usepackage[spanish,activeacute]{babel}
\usepackage{caratula}
\usepackage{a4wide}
\usepackage{hyperref}
\usepackage{fancyhdr}
\usepackage{graphicx} % Para el logo magico!
\usepackage{amssymb}
\usepackage{amsmath}
\usepackage{float}
\usepackage[latin1]{inputenc}
%\usepackage [T1]{fontenc}
\usepackage[dvipsnames,usenames]{color}
\usepackage{amsfonts}
\usepackage{ulem}
%\usepackage{highlight}
\usepackage{fancybox}
%\usepackage{marvosym}
\usepackage{color}
\usepackage{lastpage}
\usepackage{lscape}
\usepackage{tabularx}
\usepackage{algorithmic}
\usepackage{algorithm}

%%---------------------------------------------------------------------------%%
%%	SECCION opciones																												 %%	
%%---------------------------------------------------------------------------%%
\parskip    = 11 pt
\headheight	= 13.1pt
\pagestyle	{fancy}
\definecolor{orange}{rgb}{1,0.5,0}

\addtolength{\headwidth}{1.0in}

\addtolength{\oddsidemargin}{-0.5in}
\addtolength{\textwidth}{1.0in}
\addtolength{\topmargin}{-0.5in}
\addtolength{\textheight}{0.7in}

%%---------------------------------------------------------------------------%%
%%	SECCION document	 %%	
%%---------------------------------------------------------------------------%%
\begin{document}
\renewcommand{\chaptername}{Parte }
\input{trans_algorithmic.tex}
%%---- Caratula -------------------------------------------------------------%%
\materia{Problemas, Algoritmos y Programaci�n (1er cuatrimestre de 2010)}
\titulo{Trabajo Pr�ctico A}

\integrante{Gonzalez, Emiliano}{426/06}{xjesse\_jamesx@hotmail.com}
\integrante{Gonzalez, Sergio}{481/06}{gonzalezsergio2003@yahoo.com.ar}
\integrante{Tleye, Sebastian}{732/05}{s\_tleye@yahoo.com.ar}
\resumen{
En el siguiente documento, se explicar�n las soluciones encontradas a los 4 problemas
planteados para esta entrega del trabajo pr�ctico, asi tambi�n como sus modelos y algoritmos utilizados para resolverlos. Para cada uno de estos adem�s, se determinar� y demostrar� su complejidad computacional expresada en funci�n de los parametros de entrada de cada uno de los
problemas.}

% TOC, usa estilos locos
\maketitle
\pagestyle{empty}
{
\fancypagestyle{plain}
    {
    \fancyhead{}
    \fancyfoot{}
    \renewcommand{\headrulewidth}{0.0pt}
    } % clear header and footer of plain page because of ToC
\tableofcontents
}

\newpage
% arreglos los estilos para el resto del documento, y
% reseteo los numeros de pagina para que queden bien
\pagenumbering{arabic}
\fancypagestyle{plain} {
    \fancyhead[LO]{Gonzalez, Gonzalez, Tleye}
    \fancyhead[C]{}
    \fancyhead[RO]{P\'agina \thepage\ de \pageref{LastPage}}
    \fancyfoot{}
    \renewcommand{\headrulewidth}{0.4pt}
}
\pagestyle{plain}

\newpage
\chapter{Problema 1}

\section{1759 - Cubo}

\subsection{Enunciado}

En un futuro no muy lejano las personas buscar�n juegos cada vez m�s peligrosos
para jugar. Despu�s de ultra-ligeros y el bungee-jumping las personas necesitan
juegos donde sus actividades mentales tambi�n se pongan a prueba. Este es el
caso del juego llamado 'cubo', inventado en Nueva Zelanda. En algunos lugares
tambi�n es conocido por su nombre japon�s: sokoban.

Considere un laberinto de dos dimensiones compuesto de casillas cuadradas, donde
cada una est� libre o est� ocupada por una piedra. En cada paso, puede salir de
una casilla y moverse a otra casilla vecina (es decir, arriba, abajo, izquierda y
derecha) libre. Usted est� ocupando una de las casillas libres de este laberinto.

\begin{figure}[H]
\centering
\label{ej1_sokoban}
\includegraphics[scale=0.8]{./graficos/ej1/sokoban.jpg}
\caption{Ejemplo del juego}
\end{figure}

Una casilla del laberinto contiene una pila de cajas. La pila puede ser movida 
de una casilla i a una casilla k (por ejemplo, k = i+1), vecina de i, en la
direcci�n ik si usted estuviera en la casilla j (aqu� j = i-1), vecina de i, y
la direcci�n ik es igual a la direcci�n ji. Una caja no puede ser movida de
ninguna otra manera (es decir, no se puede tirar de la caja). As� que si la
caja termina en una esquina del laberinto no podr� moverla nuevamente.
Por �ltimo, tenga en cuenta que en cada empuj�n de la caja usted da un paso, y
que lo inverso no es necesariamente cierto.

Una de las casillas vac�as est� marcada como la casilla final. Su tarea es
llevar la caja a la casilla final a trav�s de una serie de pasos y empujones de
la caja. Como la caja es pesada, quiere realizar el menor n�mero posible de
empujones de la caja.

Tenga en cuenta que en el juego de la vida real existe la posibilidad de que
pueda ser aplastado por la caja, haciendo que todo mucho m�s divertido.

\textbf{Input:}

El archivo de entrada se compone de varias instancias. Cada instancia se inicia
con una l�nea que contiene dos entero r y c (ambos menores o iguales que 20) que
representando el n�mero de filas y columnas del laberinto.

Luego se les proporcionan r l�neas, cada una con c caracteres. Cada caracter
describe una casilla del laberinto. Una casilla ocupada por una piedra se indica
por \# y una casilla vac�a est� representada por un '.' (sin las comillas). Su
posici�n de partida se indica con S, la posici�n inicial de la caja est� indicada
por B y la posici�n final de la caja se indica por el T.

La entrada termina cuando r = c = 0.

\textbf{Output:}

Para cada laberinto, imprima en primer lugar el n�mero de instancia, como se
muestra en la salida de ejemplo siguiente. Si no es posible llevar el cuadro
a su posici�n final, escriba una l�nea conteniendo 'Impossivel'.
De lo contrario, deber� imprimir dos enteros x e y, donde x indica el n�mero
de movimientos (pasos + empujones) e y el numero de empujones de una secuencia
que hace que lleve la cada hasta la posici�n final. El n�mero de empujones debe
ser m�nimo. Si hay m�s de una posible secuencia que utiliza un n�mero m�nimo de
empujes, el n�mero total de movimientos debe ser m�nimo.
Imprima una l�nea en blanco despu�s de cada instancia.

\textbf{Url:}

\href{https://br.spoj.pl/problems/CUBO/}{Problema de cubo}

\subsection{Modelo}

\subsection{Soluci�n}

\subsection{Detalles de implementaci�n}

\subsection{C�lculo de complejidad}
\clearpage

\newpage
\chapter{Problema 2}

\section{Little Bishops}

Un alfil es una pieza utilizada en el juego de ajedrez el cual es jugado en una tabla con grillas cuadradas. Un alfil puede moverse solamente de forma diagonal desde su posici�n actual, y dos alfiles se atacan si uno de ellos est� en el camino del otro. En la siguiente figura, los cuadrados negros representan los lugares alcanzables por el alfil $B_1$ desde su posici�n actual. La figura tambi�n muestra que los alfiles $B_1$ y $B_2$ est�n en posiciones de ataque, pero $B_1$ y $B_3$ no. $B_2$ y $B_3$ tampoco se atacan.

\begin{figure}[H]
\centering
\label{ej2_tableroEnunciado}
\includegraphics[scale=0.5]{./graficos/ej2/tableroEnunciado.jpg}
\end{figure}

Ahora, dados dos n�meros \textbf{n} y \textbf{k}, su deber es determinar la cantidad de formas en que uno puede ubicar $k$ alfiles en un tablero de ajedrez de $n � n$, de forma tal que ningun par de ellos se atacan.
 
\textbf{Entrada:}

El archivo de entrada puede contener m�ltiples casos de test. Cada test ocupa una l�nea en el archivo de entrada y contiene dos enteros \textbf{n} $(1 \le n \le 8)$ y \textbf{k} $(0 \le k \le n^2)$. 

Un caso de test que contiene dos ceros para n y k finaliza la entrada y no necesitar� procesar esta particular entrada.
 
\textbf{Salida:}

Para cada caso de test en la entrada imprimir una l�nea conteniendo el n�mero total de formas en la cual uno puede ubicar la cantidad de alfiles dada en un tablero de ajedrez del tama�o dado tal que ningun par de ellos se atacan. Puede asumir que este n�mero ser� menor a $10^{15}$.

\textbf{Url:}

\href{http://uva.onlinejudge.org/index.php?option=com\_onlinejudge\&Itemid=8\&category=10\&page=show\_problem\&problem=802}{Problema de Little Bishops}

\subsection{Soluci�n}

Para la soluci�n de este problema, se tuvo como idea de resoluci�n una t�cnica de backtracking. La primera aproximaci�n a la soluci�n que surgi� fue simplemente ir ubicando los alfiles uno por uno, en todas las combinaciones de posiciones (sin discriminar alfiles, es decir, la solucion en la que los alfiles $B_1$ y $B_2$ est�n ubicados en ciertas casillas es la misma soluci�n con los alfiles permutados, por lo tanto solo consideramos una de ellas), podando aquellas ramas en las que dos alfiles se atacan entre s�. Si bien esta aproximaci�n es correcta, tardaba mucho tiempo en devolver la soluci�n.

Utilizando esta aproximaci�n a la soluci�n, tenemos por fin disminu�r el tiempo de ejecuci�n del algoritmo. Para esto vamos a aprovechar una propiedad que es particular de los alfiles: si un alfil est� posicionado en un casillero blanco entonces es imposible que se ataque con un alfil ubicado en un casillero negro. Entonces podemos ahorrar c�lculos, dividiendo el problema en dos casos:

\begin{itemize}
\item $n$ es par: en este caso la cantidad de casilleros blancos es igual a la cantidad de casilleros negros. Por lo tanto la cantidad de formas en que uno puede ubicar $k_c$ alfiles es igual tanto para los casilleros blancos como para los negros. Llamemos $f:Nat�Nat \rightarrow Nat$ a la funci�n tal que dada una cantidad $k_c$ de alfiles disponibles para ubicar en un color, y un $n$ que se corresponde con el n de la entrada, $f(n, k_c)$ devuelve la cantidad de formas en las que se puede ubicar $k_c$ alfiles en un solo color de un tablero de $n�n$. Entonces la cantidad total de formas en las que se pueden ubicar $k$ alfiles en un tablero de ajedrez de $n�n$, con $n$ par es

\medskip
\medskip
\centerline{ $cantSoluciones(n, k) = \sum_{i = 0}^k f(n, i) * f(n, k - i)$ }
\medskip
\medskip

pues por cada forma de ubicar $i$ alfiles en un color, tenemos $f(n, k - i)$ formas de ubicar $k - i$ alfiles en el otro color.

Entonces el algoritmo calcula $f(n, i)$ para $i = 0, ..., k$ haciendo backtracking como en la primera aproximaci�n, y realiza la sumatoria.

\item $n$ es impar: en este caso la cantidad de casilleros blancos difiere en uno con la cantidad de casilleros negros. Usamos un c�lculo similar al anterior. Llamemos $f_B:Nat�Nat \rightarrow Nat$ y $f_N:Nat�Nat \rightarrow Nat$ a las funci�nes tales que dado un $n$ que se corresponde con el n de la entrada y una cantidad $k_c$ de alfiles disponibles para ubicar en un color, $f_B(n, k_c)$ y $f_N(n, k_c)$ devuelven la cantidad de formas en las que se puede ubicar $k_c$ alfiles en el color blanco y en el color negro, respectivamente, ambos para un tablero de $n�n$ casilleros. Entonces la cantidad total de formas en las que se pueden ubicar $k$ alfiles en un tablero de ajedrez de $n�n$, con $n$ impar es

\medskip
\medskip
\centerline{ $cantSoluciones(n, k) = \sum_{i = 0}^k f_B(n, i) * f_N(n, k - i)$ }
\medskip
\medskip

pues por cada forma de ubicar $i$ alfiles en el color blanco, tenemos $f_N(n, k - i)$ formas de ubicar $k - i$ alfiles en el color negro.

Entonces el algoritmo calcula $f_B(n, i)$ y $f_N(n, i)$ para $i = 0, ..., k$ haciendo backtracking como en la primera aproximaci�n, y realiza la sumatoria.
\end{itemize}

Por �ltimo, para evitar hacer c�lculos, sabemos que si $k > 2 * ( n - 1 )$ para $n > 1$ entonces la cantidad de formas en las que se pueden ubicar $k$ alfiles en un tablero de $n�n$ es \textbf{0}, pues en todo tablero de $n�n$ hay $2 * ( 2 * n - 1 )$ diagonales. Cada alfil ocupa dos diagonales, por lo tanto luego de ubicar el alfil n�mero $2 * ( n - 1 )$ no hay forma de posicionar el siguiente alfil sin que se ataque con otro. Por lo tanto en este caso el algoritmo devuelve directamente \textbf{0}.

A modo de aclaraci�n, veamos unos ejemplos de c�mo funciona el algoritmo:

Supongamos que tenemos un tablero de $2�2$ y debemos ubicar 2 alfiles:

\begin{figure}[H]
\centering
\label{ej2_tablero2x2_0}
\includegraphics[scale=0.5]{./graficos/ej2/tablero2x2_0.png}\hspace{0.5in} 
\includegraphics[scale=0.5]{./graficos/ej2/alfil.png}\hspace{0.5in} 
\includegraphics[scale=0.5]{./graficos/ej2/alfil.png}\hspace{0.5in} 
\caption{Tablero de ajedrez con n = 2 y k = 2}
\end{figure}

Por supuesto no se pueden ubicar 2 alfiles en el mismo color, por lo tanto $f(2, 2) = 0$, y las formas de ubicar 0 alfiles en un color es $f(2, 0) = 1$.

Veamos que pasa con $f(2, 1)$:

\begin{figure}[H]
\centering
\label{ej2_tablero2x2_1,2}
\includegraphics[scale=0.5]{./graficos/ej2/tablero2x2_1.png}\hspace{0.5in}
\includegraphics[scale=0.5]{./graficos/ej2/tablero2x2_2.png}\hspace{0.5in} 
\caption{Ubicaciones de 1 alfil en un tablero de 2x2, para el color negro}
\end{figure}

Y como vemos en el siguiente grafico, la cantidad de formas de colocar un alfil en el color blanco es la misma que para el color negro, como explicamos antes

\begin{figure}[H]
\centering
\label{ej2_tablero2x2_3,4}
\includegraphics[scale=0.5]{./graficos/ej2/tablero2x2_3.png}\hspace{0.5in}
\includegraphics[scale=0.5]{./graficos/ej2/tablero2x2_4.png}\hspace{0.5in} 
\caption{Ubicaciones de 1 alfil en un tablero de 2x2, para el color blanco}
\end{figure}

Por lo tanto nuestro algoritmo har� el c�lculo 

\medskip
\centerline{ $cantSoluciones(2, 2) = f(2, 0) * f(2, 2) + f(2, 1) * f(2, 1) + f(2, 2) * f(2, 0)$ }
\centerline{ $cantSoluciones(2, 2) = 1 * 0 + 2 * 2 + 0 * 1$ }
\centerline{ $cantSoluciones(2, 2) = 4$ }
\medskip

Entonces la cantidad de formas de colocar 2 alfiles en un tablero de $2�2$ son \textbf{4}, que son:

\begin{figure}[H]
\centering
\label{ej2_tablero2x2_5-8}
\includegraphics[scale=0.5]{./graficos/ej2/tablero2x2_5.png}\hspace{0.5in}
\includegraphics[scale=0.5]{./graficos/ej2/tablero2x2_6.png}\hspace{0.5in} 
\includegraphics[scale=0.5]{./graficos/ej2/tablero2x2_7.png}\hspace{0.5in}
\includegraphics[scale=0.5]{./graficos/ej2/tablero2x2_8.png}\hspace{0.5in} 
\caption{Ubicaciones de 2 alfiles en un tablero de 2x2}
\end{figure}

Supongamos ahora que tenemos un tablero de $3�3$ y queremos ubicar 3 alfiles:

\begin{figure}[H]
\centering
\label{ej2_tablero3x3_0}
\includegraphics[scale=0.5]{./graficos/ej2/tablero3x3_00.png}\hspace{0.5in} 
\includegraphics[scale=0.5]{./graficos/ej2/alfil.png}\hspace{0.5in} 
\includegraphics[scale=0.5]{./graficos/ej2/alfil.png}\hspace{0.5in} 
\includegraphics[scale=0.5]{./graficos/ej2/alfil.png}\hspace{0.5in} 
\caption{Tablero de ajedrez con n = 3 y k = 3}
\end{figure}

Nuestro algoritmo har� el siguiente c�lculo:

\medskip
\centerline{ $cantSoluciones(3, 3) = f_N(3, 0) * f_B(3, 3) + f_N(3, 1) * f_B(3, 2) + f_N(3, 2) * f_B(3, 1) + f_N(3, 3) * f_B(3, 0)$ }
\centerline{ $cantSoluciones(3, 3) = 1 * 0 + 4 * 4 + 5 * 2 + 1 * 0$ }
\centerline{ $cantSoluciones(3, 3) = 26$ }
\medskip

Entonces la cantidad de formas de colocar 3 alfiles en un tablero de $3�3$ son \textbf{26}, que son:

\begin{figure}[H]
\centering
\label{ej2_tablero3x3_1-26}
\includegraphics[scale=0.2]{./graficos/ej2/tablero3x3_01.png}\hspace{0.5in}
\includegraphics[scale=0.2]{./graficos/ej2/tablero3x3_02.png}\hspace{0.5in} 
\includegraphics[scale=0.2]{./graficos/ej2/tablero3x3_03.png}\hspace{0.5in}
\includegraphics[scale=0.2]{./graficos/ej2/tablero3x3_04.png}\hspace{0.5in} 
\includegraphics[scale=0.2]{./graficos/ej2/tablero3x3_05.png}\hspace{0.5in} 
\includegraphics[scale=0.2]{./graficos/ej2/tablero3x3_06.png}\hspace{0.5in} 
\includegraphics[scale=0.2]{./graficos/ej2/tablero3x3_07.png}\hspace{0.5in} 
\includegraphics[scale=0.2]{./graficos/ej2/tablero3x3_08.png}\hspace{0.5in} 
\includegraphics[scale=0.2]{./graficos/ej2/tablero3x3_09.png}\hspace{0.5in} 
\includegraphics[scale=0.2]{./graficos/ej2/tablero3x3_10.png}\hspace{0.5in} 
\includegraphics[scale=0.2]{./graficos/ej2/tablero3x3_11.png}\hspace{0.5in} 
\includegraphics[scale=0.2]{./graficos/ej2/tablero3x3_12.png}\hspace{0.5in} 
\includegraphics[scale=0.2]{./graficos/ej2/tablero3x3_13.png}\hspace{0.5in} 
\includegraphics[scale=0.2]{./graficos/ej2/tablero3x3_14.png}\hspace{0.5in} 
\includegraphics[scale=0.2]{./graficos/ej2/tablero3x3_15.png}\hspace{0.5in} 
\includegraphics[scale=0.2]{./graficos/ej2/tablero3x3_16.png}\hspace{0.5in} 
\includegraphics[scale=0.2]{./graficos/ej2/tablero3x3_17.png}\hspace{0.5in} 
\includegraphics[scale=0.2]{./graficos/ej2/tablero3x3_18.png}\hspace{0.5in} 
\includegraphics[scale=0.2]{./graficos/ej2/tablero3x3_19.png}\hspace{0.5in} 
\includegraphics[scale=0.2]{./graficos/ej2/tablero3x3_20.png}\hspace{0.5in} 
\includegraphics[scale=0.2]{./graficos/ej2/tablero3x3_21.png}\hspace{0.5in} 
\includegraphics[scale=0.2]{./graficos/ej2/tablero3x3_22.png}\hspace{0.5in} 
\includegraphics[scale=0.2]{./graficos/ej2/tablero3x3_23.png}\hspace{0.5in} 
\includegraphics[scale=0.2]{./graficos/ej2/tablero3x3_24.png}\hspace{0.5in} 
\includegraphics[scale=0.2]{./graficos/ej2/tablero3x3_25.png}\hspace{0.5in} 
\includegraphics[scale=0.2]{./graficos/ej2/tablero3x3_26.png}\hspace{0.5in} 
\caption{Ubicaciones de 3 alfiles en un tablero de 3x3}
\end{figure}

\subsection{Detalles de implementaci�n}
El tablero no est� explicitamente implementado en ninguna estructura. En su lugar utilizamos dos vectores booleanos, uno para las diagonales ascendentes y otro para las diagonales descendentes. Cada posici�n del vector es una diagonal y esta en true si la misma est� ocupada y en false si no.

Para decidir si una posici�n es segura (es decir, no es posici�n de ataque), se mantienen dos enteros correspondientes a la fila y columna de la posici�n en cuesti�n, y se indexa los vectores de diagonales ascendentes y descendentes con $fila + columna$ y $fila - columna + n - 1$ respectivamente. Es decir, la diagonal ascendente en la posici�n superior izquierda se corresponde con la posici�n $0$ del vector de diagonales ascendentes, y la diagonal ascendente en la posici�n inferior derecha se corrresponde con la posici�n $2 * n - 1$; la diagonal descendente en la posici�n superior derecha se corresponde con la posici�n $0$ del vector de diagonales descendentes, y la diagonal descendente en la posici�n inferior izquierda se corresponde con la posici�n $2 * n - 1$.

En el modelo explicado describimos dos funciones principales: \textbf{cantSoluciones} y \textbf{f}. La primera se corresponde con \textbf{cantSoluciones} de la implementaci�n, y la segunda con \textbf{solucionesUnColor}. Notar que solucionesUnColor, adem�s de recibir por par�metro el $n$ y la cantidad de alfiles, recibe el color donde queremos ubicarlo, de modo que $solucionesUnColor(n, alfiles, Negro)$ es $f_N(n,alfiles)$ y $solucionesUnColor(n, alfiles, Blanco)$ es $f_B(n,alfiles)$.

\subsection{C�lculo de complejidad}

En esta secci�n se calcular� la complejidad del algortimo que calcula la soluci�n. Se har� en base a un pseudoc�digo de las dos funciones principales: \textbf{cantSoluciones} y \textbf{solucionesUnColor}.

A continuaci�n, el pseudoc�digo de la funci�n cantSoluciones:

\begin{algorithm}[H]
\caption{Calcula la cantidad de formas de ubicar k alfiles en un tablero de nxn}
\label{alg:algoritmo2_1}
\begin{algorithmic}[1]
\PARAMS{n, k}
\IF{$k > 2 * ( n - 1 ) y n > 1$}
	\RETURN $0$
\ELSE
	\STATE inicializar cantSoluciones en $0$
	\IF{n es par}
		\STATE \COMMENT{En la pr�xima l�nea es lo mismo calcular las soluciones para el color negro o para el blanco, por lo tanto elegimos negro de forma arbitraria}
		\RETURN sumatoria de $solucionesUnColor(n, alfiles, color) * solucionesUnColor(n, k - alfiles, color)$, para $c = Negro$ y $alfiles = 0 ... k$
	\ELSE
		\RETURN sumatoria de $solucionesUnColor(n, alfiles, Negro) * solucionesUnColor(n, k - alfiles, Blanco)$, para $alfiles = 0 ... k$
	\ENDIF
\ENDIF
\end{algorithmic}
\end{algorithm}

A continuaci�n, el pseudoc�digo de la funci�n solucionesUnColor:

\begin{algorithm}[H]
\caption{Calcula la cantidad de formas de ubicar un n�mero de alfiles en un tablero de nxn, en el color c}
\label{alg:algoritmo2_2}
\begin{algorithmic}[1]
\PARAMS{n, alfiles, c}
\IF{$alfiles == 0$}
	\RETURN $1$
\ELSE
	\STATE cantSoluciones $\leftarrow$ 0
	\STATE \COMMENT{La primera llamada a la funci�n supone no haber recorrido ninguna casilla, por lo que en la siguiente l�nea comenzar� a iterar por la casilla superior izquierda}
	\FOR{cada casilla del color c a partir de la siguiente en la que estaba al llamar esta funci�n}
		\IF{la casilla es segura}
			\STATE ubicar un alfil en la casilla
			\STATE cantSoluciones $\leftarrow$ cantSoluciones $+$ solucionesUnColor($n, alfiles - 1, c$)
			\STATE quitar el alfil de la casilla
		\ENDIF
	\ENDFOR
	\RETURN cantSoluciones
\ENDIF
\end{algorithmic}
\end{algorithm}
\clearpage

\newpage
\chapter{Problema 3}

\section{10594 - Data Flow}

\textbf{Entrada:}

\textbf{Salida:}

\textbf{Url:}

\href{http://uva.onlinejudge.org/index.php?option=com_onlinejudge&Itemid=8&category=17&page=show_problem&problem=1535}{Problema de data flow}

\subsection{Modelo}

\subsection{Soluci�n}

\subsection{Detalles de implementaci�n}

\subsection{C�lculo de complejidad}
\clearpage

\newpage
\chapter{Problema 4}

\section{Minimal Coverage}

Dados segmentos de una l�nea (en el eje X) con coordenadas [$L_i$,$R_i$]. Se debe elegir la minima cantidad de estos segmentos, tales que cubran completamente el segmento [$0$,$M$].

\textbf{Input:}

El input esta dividido en varias partes:

La primera l�nea, consiste en un n�mero e indica la cantidad de casos de test, seguido de una l�nea en blanco.

Luego sigue el test, que consiste en un n�mero, que identifica el segmento [$0$,$M$] (donde $1<= M <= 5000$), seguido de pares $L_i$ $R_i$ que identifican segmentos ($|L_i|$, $|R_i|$ $<=50000$, e $i <= 100000$), cada uno separados por una linea. Cada test termina con el par $0$ $0$.

\textbf{Output:}

Para cada caso de test, la primera l�nea indica el m�nimo numero de segmentos que pueden cubrir el segmento [$0$,$M$]. En las l�neas siguientes, las coordenadas de los segmentos, ordenados por su coordenada izquierda ($L_i$), con el mismo formato que en el input. El par $0$ $0$ no debe ser mostrado.

Si [$0$,$M$] no puede ser cubierto, se debe devolver 0.

Cada resultado entre tests, debe estar separado por una l�nea en blanco.

\textbf{Url:}

\href{http://uva.onlinejudge.org/index.php?option=com\_onlinejudge\&Itemid=8\&category=12\&page=show\_problem\&problem=961}{Problema de minimal coverage}

\subsection{Soluci�n}

Para llegar a la soluci�n de este problema, se tuvo como idea de resoluci�n una t�cnica golosa, donde en cada paso se decide cuales de las opciones que se tienen disponibles en ese paso, garantizan un resultado �ptimo.

Dicha idea consiste b�sicamente, en ir tomando los intervalos que cubran mas a la derecha posible desde el intervalo cubierto hasta el momento y luego, una vez encontrado dicho intervalo, actualizar la zona cubierta hasta ese paso. Para hacer esto, como primera medida, se ordenan los intervalos recibidos como entrada del problema, de menor a mayor, por su primera coordenada. Con esta medida se pueden ir recorriendo en orden los intervalos de entrada y escoger aquel que nos pueda ayudar a dar un soluci�n optima.

A continuaci�n se describir�n los pasos mas importantes de la soluci�n para que la idea quede mejor explicada.

\begin{enumerate}
\item Paso 1: En este paso, se ordenan los intervalos recibidos como entrada del problema y se los ordena de menor a mayor por su primera componente. Esto se hace para facilitar la b�squeda del mejor intervalo en cada paso.
\item Paso 2: Utilizando la lista de pares ordenados del punto anterior, lo que se hace aqui, es recorrer cada par (intervalo) de forma tal de seleccionar aquel que tenga su segunda coordenada mas grande y que comience dentro del intervalo cubierto hasta el momento (es decir que su primera coordenada sea menor a cierto valor, que en principio es 0, ya que no se ha cubierto nada a�n).
\item Paso 3: Una vez obtenido el intervalo del paso 2, se actualiza el valor que representa lo cubierto hasta el momento, cambi�ndolo por la segunda coordenada del intervalo obtenido.
\end{enumerate}

A continuaci�n se mostrar� como el m�todo encuentra la soluci�n para un ejemplo, en donde se ingresan los intervalos [1,3], [-2,0], [0,1], [-1,2] y donde el intervalo a cubrir es el [0,3]. 


\begin{figure}[H]
\centering
\label{ej4_1}
\subfigure[intervalos en el orden en que fueron ingresados.]{
\includegraphics[scale=0.5]{./graficos/ej4/ej4_1_0.png} }\hspace{0.5in} 
\subfigure[intervalos una vez ordenados.]{
\includegraphics[scale=0.5]{./graficos/ej4/ej4_1_1.png}}\hspace{0.5in} 
\subfigure[seleccionar los intervalos que tengan alguna coordenada en el intervalo actual $(0,0)$]{
\includegraphics[scale=0.5]{./graficos/ej4/ej4_1_2.png}}\hspace{0.5in} 
\subfigure[se selecciona el que tiene la segunda coordenada mayor (intervalo b), se lo agrega como soluci�n, y se actualiza el intervalo cubierto, que ser� $(0,2)$]{
\includegraphics[scale=0.5]{./graficos/ej4/ej4_1_3.png}} \hspace{0.5in} 
\subfigure[se seleccionan los intervalos siguientes y que tengan alguna coordenada dentro del intervalo actual $(0,2)$]{
\includegraphics[scale=0.5]{./graficos/ej4/ej4_1_4.png} }\hspace{0.5in} 
\subfigure[se toma el de mayor segunda coordenada(intervalo d), se agrega como soluci�n y se actualiza el intervalo cubierto a $(0,3)$]{
\includegraphics[scale=0.5]{./graficos/ej4/ej4_1_5.png}}
\caption{Pasos hasta hallar la soluci�n}
\setcounter{subfigure}{0}
\end{figure}

\subsection{Por qu� esta soluci�n?}

En cuanto al criterio de orden entre intervalos al momento de ordenarlos, elegimos que sea por su primera coordenada, pues esto nos permite avanzar (es decir, iterativamente agrandar el intervalo cubierto para llegar a la soluci�n) linealmente en el intervalo [0,M], lo que no es posible si el orden fuese por segunda coordenada. De usar un criterio de ordenamiento como este segundo, ser�a conveniente avanzar en el intervalo [0,M] desde fin a principio.

Ordenar los intervalos es escencialmente innecesario para el problema, pues para construir una soluci�n se podr�a recorrer siempre todos los intervalos y avanzar en [0,M] elegiendo seg�n el criterio del paso 2 del algoritmo. Sin embargo ordenar los intervalos nos asegura una mejor complejidad en peor caso.

El paso 2 nos asegura que no elegiremos intervalos de m�s y que el algoritmo siempre avanza y nunca vuelve en el intervalo ya cubierto (recorre linealmente, sin volver atras, la lista de intervalos que a este punto est� ordenada, y va construyendo la soluci�n). Al considerar aquellos pares que tengan su primera coordenada dentro del intervalo ya cubierto evitamos dejar �huecos� en el desarrollo de la soluci�n para no volver entre los intervalos ya vistos. Al elegir entre ellos aquel que tenga su mayor segunda coordenada logramos avanzar lo m�ximo posible en cada iteraci�n, sin perder soluciones y asegurando siempre la de menor cantidad de intervalos. Veamos de nuevo el ejemplo de la figura \ref{ej4_1}. Si hubi�semos elegido el intervalo c en lugar del b, la cantidad de intervalos final ser�a igual (2 intervalos). Sin embargo esto no siempre es as�. A modo de aclaraci�n, veamos los casos que se pueden presentar al momento de elegir los intervalos que forman la soluci�n.

Sea [x,y] un intervalo a cubrir, y sean los siguientes intervalos:

\begin{figure}[H]
\centering
\label{ej4_2_0}
\subfigure[intervalo a cubrir $(x,y)$ e intervalos de entrada]{
\includegraphics[scale=0.5]{./graficos/ej4/ej4_2_0.png}}\hspace{0.5in} 
\setcounter{subfigure}{0}
\end{figure}

Supongamos que tenemos un algoritmo alternativo \textbf{MC}, que en lugar de elegir el intervalo que m�s cubre (que comienza en el intervalo ya cubierto y tiene la mayor segunda coordenada), selecciona cualquier intervalo al azar entre los que comienzan en el intervalo ya cubierto y terminan fuera. Compararemos la soluci�n que encuentra MC y la que encuentra nuestro algoritmo:

\begin{figure}[H]
\centering
\label{ej4_2_1,2}
\subfigure[soluci�n que encuentra nuestro algoritmo]{
\includegraphics[scale=0.5]{./graficos/ej4/ej4_2_1.png}} \hspace{0.5in} 
\subfigure[soluci�n que encuentra MC]{
\includegraphics[scale=0.5]{./graficos/ej4/ej4_2_2.png} }\hspace{0.5in} 
\caption{Soluci�n que encuentra MC, de igual cantidad de intervalos}
\setcounter{subfigure}{0}
\end{figure}

Como vemos, la soluci�n alternativa requiere la misma cantidad de intervalos que los que nuestro algoritmo calcula, por lo tanto para este caso nuestro algoritmo encuentra igual cantidad de intervalos para cubrir [x,y] que MC.

Veamos otro caso. Nuevamente tenemos el intervalo a cubrir [x,y] y los intervalos de entrada:

\begin{figure}[H]
\centering
\label{ej4_3_0}
\subfigure[intervalo a cubrir $(x,y)$ e intervalos de entrada]{
\includegraphics[scale=0.5]{./graficos/ej4/ej4_3_0.png}}\hspace{0.5in} 
\setcounter{subfigure}{0}
\end{figure}

Las soluciones correspondientes a nuestro algoritmo y a MC son:

\begin{figure}[H]
\centering
\label{ej4_3_1,2}
\subfigure[soluci�n que encuentra nuestro algoritmo]{
\includegraphics[scale=0.5]{./graficos/ej4/ej4_3_1.png}} \hspace{0.5in} 
\subfigure[soluci�n que encuentra MC]{
\includegraphics[scale=0.5]{./graficos/ej4/ej4_3_2.png} }\hspace{0.5in} 
\caption{Soluci�n que encuentra MC, de distinta cantidad de intervalos}
\setcounter{subfigure}{0}
\end{figure}

La soluci�n que nuestro algoritmo encuentra es la �ptima, sin embargo MC encuentra una soluci�n con 3 intervalos. Esto es as� porque cuando estaba eligiendo entre los intervalos a y b, MC incluy� a b en su soluci�n, y sigui� adelante con c y d (recordar que no se vuelve a decidir con intervalos ya antes vistos, pues se recorre la lista de intervalos de forma lineal y se va armando la soluci�n).

Veamos otro caso. Nuevamente tenemos el intervalo a cubrir [x,y] y los intervalos de entrada:

\begin{figure}[H]
\centering
\label{ej4_4_0}
\subfigure[intervalo a cubrir $(x,y)$ e intervalos de entrada]{
\includegraphics[scale=0.5]{./graficos/ej4/ej4_4_0.png}}\hspace{0.5in} 
\setcounter{subfigure}{0}
\end{figure}

Las soluciones correspondientes a nuestro algoritmo y a MC son:

\begin{figure}[H]
\centering
\label{ej4_4_1,2}
\subfigure[soluci�n que encuentra nuestro algoritmo]{
\includegraphics[scale=0.5]{./graficos/ej4/ej4_4_1.png}} \hspace{0.5in} 
\subfigure[soluci�n que encuentra MC]{
\includegraphics[scale=0.5]{./graficos/ej4/ej4_4_2.png} }\hspace{0.5in} 
\caption{Nuestro algoritmo encuentra soluci�n, pero MC no}
\setcounter{subfigure}{0}
\end{figure}

En el primer paso MC ignora el intervalo que m�s cubre en la primera iteraci�n, que es a, y elige b. En el segundo paso, como no vuelve a elegir entre intervalos ya descartados (a est� descartado), nota que no hay ning�n intervalo que comience en el intervalo ya cubierto (que en el segundo paso el intervalo cubierto es [x, b2]) y termine fuera, por lo tanto no encuentra soluci�n. Vemos entonces que nuestro algoritmo encontr� una mejor soluci�n.

Por �ltimo, tenemos otro caso:

\begin{figure}[H]
\centering
\label{ej4_5_0}
\subfigure[intervalo a cubrir $(x,y)$ e intervalos de entrada]{
\includegraphics[scale=0.5]{./graficos/ej4/ej4_5_0.png}}\hspace{0.5in} 
\setcounter{subfigure}{0}
\end{figure}

Las soluciones correspondientes a nuestro algoritmo y a MC son:

\begin{figure}[H]
\centering
\label{ej4_5_1,2}
\subfigure[soluci�n que encuentra nuestro algoritmo]{
\includegraphics[scale=0.5]{./graficos/ej4/ej4_5_1.png}} \hspace{0.5in} 
\subfigure[soluci�n que encuentra MC]{
\includegraphics[scale=0.5]{./graficos/ej4/ej4_5_2.png} }\hspace{0.5in} 
\caption{Problema sin soluci�n}
\setcounter{subfigure}{0}
\end{figure}

Ninguno de los dos algoritmos encuentra soluci�n, pues no la hay. No se puede cubrir el intervalo [x,y] en este caso.

En conclusi�n, nuestro algoritmo es igual o mejor que los demas (en cuanto a encontrar la m�nima cantidad de intervalos que cubre [0,M]).

El paso 3 nos asegura que el algoritmo termina, y que lo hace al cubrir el intervalo [0,M] o al encontrar un hueco dentro del mismo con los intervalos dados. Si se encuentra que el intervalo cubierto hasta el momento es menor que M, y que no hay ningun par que comience en el intervalo cubierto y que termine fuera, entonces el algoritmo devuelve la soluci�n sin intervalos. Si se llega a un intervalo cubierto mayor a M, entonces termina y devuelve los pares que lo cubren.

\subsection{Detalles de implementaci�n}
Utilizamos \textbf{Vector} de la \textbf{Standart Template Library (STL)} de \textbf{C++} para representar la secuencia de intervalos, y creamos una propia clase \textbf{Intervalo} que es b�sicamente una tupla de enteros, que especifica los m�todos de comparaci�n entre ellos para poder realizar el ordenamiento y otros c�lculos.

Adem�s creamos una clase \textbf{TestCase} para tratar cada caso de test de la entrada por separado. El algoritmo que resuelve el problema est� todo inclu�do dentro del m�todo llamado \"resolver\" de esta clase.

La lectura del input y escritura del output ser�n obviados pues no forman parte de la escencia del problema.

\subsection{C�lculo de complejidad}

\begin{algorithm}[H]
\caption{Calcula el cubrimiento minimal}
\label{alg:algoritmo1b}
\begin{algorithmic}[1]
\PARAMS{M, Intervalos}
\STATE intervaloCubierto $\leftarrow$ 0
\STATE inicializar resultado como lista de intervalos vac�a
\STATE ordenar los intervalos por su primera coordenada
\STATE ignorar los primeros pares de la lista de intervalos que tienen su segunda coordenada menor a 0
\FOR{cada intervalo I}
	\IF{intervaloCubierto $>$ M}
		\RETURN resultado
	\ENDIF
	\IF{la primer coordenada de I es mayor a intervaloCubierto}
		\RETURN vac�o
	\ELSE
		\IF{I es el de mayor segunda coordenada entre los que comienzan en el intervalo ya cubierto}
			\STATE resultado.push\_back(I)
			\STATE intervaloCubierto $\leftarrow$ segunda coordenada de I
		\ENDIF
	\ENDIF
\ENDFOR
\IF{intervaloCubierto $<$ M}
	\RETURN vac�o
\ENDIF
\RETURN resultado
\end{algorithmic}
\end{algorithm}

Primero definimos algunas cuestiones con respecto a los s�mbolos que se utilizar�n para demostrar la complejidad. El enunciado dice que se recibe una lista de casos de test, que contienen una lista de intervalos y el M, y para cada test, a partir de esta lista nuestro algoritmo calcular� la salida. Llamamos $N$ a la cantidad de intervalos de la entrada para un caso de test.

Las primeras dos l�neas son inicializaciones de variables, donde en la l�nea 1 se inicializa el intervalo cubierto a 0, es decir que si bien haciendo esto estamos considerando que ya cubrimos el intervalo $[0,0]$, luego en la l�nea 2 inicializamos el resultado (que es la lista de intervalos de la soluci�n) en vac�o, por lo que al empezar ning�n intervalo forma parte de la soluci�n.

La l�nea 3 ordena la lista de intervalos de entrada por su primera componente, lo que, utilizando el sort de la \textbf{Standart Template Library (STL)}, se puede lograr en aproximadamente $N*log(N)$ comparaciones en caso promedio y $N^2$ como peor caso.

A partir de la l�nea 4, se recorre linealmente la lista de intervalos de entrada. La l�nea 4 ignora aquellos que no sirvan para la soluci�n, avanzando el iterador.

En la l�nea 5 comienza un ciclo donde se recorren todos los intervalos (o hasta que se cubre $[0,M]$), empezando desde donde termin� la l�nea 4. Entonces dicho ciclo y la l�nea 4 juntos se ejecutar�n tantas veces como intervalos que haya (en peor caso), es decir, $N$ veces.

Analicemos ahora los pasos dentro del ciclo. Las l�neas 6 hasta la 11 tienen costo constante, pues se trata de comparaciones entre enteros.

La l�nea 12 resume un ciclo que recorre la secuencia de intervalos y actualiza en cada paso el mejor par hasta el momento (aquel que comience en el intervalo ya cubierto y tenga mayor segunda coordenada). En realidad esto no se hace por cada intervalo, sino que este mismo ciclo se ocupa de avanzar en la lista hasta que el actual intervalo tenga su primera coordenada mayor al intervalo cubierto, es decir, que entre este par y el intervalo cubierto hay un hueco. El costo de avanzar en la secuencia de intervalos esta inclu�do en el costo de la l�nea 5. Las comparaciones y asignaciones de esta l�nea tambi�n se pueden lograr en costo constante, pues se trata simplemente de guardar un puntero a un intervalo y actualizar tal puntero y realizar operaciones entre enteros para comparar intervalos.

La l�nea 13 que corresponde a la inserci�n (detr�s) de un vector tiene costo constante si obviamos que los vectores pueden llegar a reajustar su tama�o (debido a la implementaci�n de STL). Esto se puede evitar, por ejemplo, ajustando su tama�o previamente de forma conveniente.

La l�nea 14 tiene costo constante por tratarse de una asignaci�n de enteros.

En la l�nea 17 finaliza el ciclo que itera los intervalos, por lo que a continuaci�n, las siguientes l�neas no ser�n afectadas por el factor $N$ que implica recorrer la secuencia.

La l�nea 18 tiene costo constante por tratarse de una comparaci�n entre enteros.

La l�nea 19 tiene costo lineal $N$ por tratarse de limpiar el vector de resultados y devolver el mismo. Lo hacemos con el m�todo clear de vector de STL.

La l�nea 21 tiene costo constante pues solamente se devuelve el resultado.

Con lo explicado anteriormente podemos conclu�r que el costo en peor caso ser� de 

\medskip
\centerline{ $N^2 + 2N$ }

o sea

\medskip
\centerline{ $N^2$ }

y en caso promedio ser� de

\medskip
\centerline{ $N*log( N ) + 2N$ }

o sea

\medskip
\centerline{ $N*log( N ) + 2N$ }

por lo tanto tiene una complejidad en peor caso de $O( N^2 )$ y en caso promedio $O( N*log( N ) )$.

\clearpage

\label{LastPage}
\end{document}
